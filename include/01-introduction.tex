%%%%%%%%%%%%%%%%%%%%%%%%%%%%%%%%%%%%%%%%%%%%%%%%%%%%%%%%%%%%%%%%%%%%%%%%%%%%%%%
%
% Introduction
% 
%%%%%%%%%%%%%%%%%%%%%%%%%%%%%%%%%%%%%%%%%%%%%%%%%%%%%%%%%%%%%%%%%%%%%%%%%%%%%%%


\chapter{Introduction}
Fluid-Structure Interaction (FSI) problems describe the coupled dynamics of fluid mechanics and structure mechanics. They are classical multi-physics problems and its application is very vast. Numerical simulation of such problems is a cumbersome process. There are mainly two approaches in solving the FSI problems \textit{Monolithic approach} and \textit{Partitioned approach}. In the current study, the focus is on the partitioned approach for solving FSI problems.
 


\section{Motivation}

Specific motivation for the problem at hand. 


\section{Task}

Concrete task to be solved. 



\section{Related Work}

Other relevant academic work and how it differs from this work, for
example \citet{shannon_diff} and \citet{blowfish}. Distinguish between
``textual'' citation, as shown in \citet{shannon_diff}, and
``parenthesis'' citation \citep{blowfish}.



\section{Results}

What has been achieved in this work? 


\section{Outline}

How is the thesis structured and why? 


\section{Acknowledgments}

A big thank you for the support to \ldots 

