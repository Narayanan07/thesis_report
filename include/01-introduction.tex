%%%%%%%%%%%%%%%%%%%%%%%%%%%%%%%%%%%%%%%%%%%%%%%%%%%%%%%%%%%%%%%%%%%%%%%%%%%%%%%
%
% Introduction
% 
%%%%%%%%%%%%%%%%%%%%%%%%%%%%%%%%%%%%%%%%%%%%%%%%%%%%%%%%%%%%%%%%%%%%%%%%%%%%%%%


\chapter{Introduction}
Fluid-Structure Interaction (FSI) problems describe the coupled dynamics of fluid mechanics and structure mechanics. They are classical multi-physics problems and its application is very vast. Numerical simulation of such problems is a cumbersome process. There are mainly two approaches in solving the FSI problems \textit{Monolithic approach} and \textit{Partitioned approach}. In the current study, the focus is on the partitioned approach for solving FSI problems. Partitioned analysis techniques are more popular than the fully coupled Monolithic solvers, as they have computational superiority over the Monolithic solvers. Partitioned solvers allows for the use of suitable discretization methods, and optimized solvers for modeling of both fluid and structure. In the current investigation, adaptive schemes such as \textit{Aitken's $\Delta^2$ method} and \textit{Steepest descent/gradient method} are implemented into the FSI solver to predict the \textit{under-relaxation factor} dynamically.  The current study also involves integration of an \textit{Artificial Neural Network (ANN)} within the FSI solver for dynamic prediction of under-relaxation factors. The calculations are performed on an in-house \textit{finite volume} based \textit{FORTRAN} solver, \textit{FASTEST-3D}. A partitioned, semi-implicit predictor-corrector coupling scheme method is used for solving FSI problems. 

\section{Need for dynamic relaxation methods}


\section{Task}

Concrete task to be solved. 



\section{Related Work}

Other relevant academic work and how it differs from this work, for
example \citet{shannon_diff} and \citet{blowfish}. Distinguish between
``textual'' citation, as shown in \citet{shannon_diff}, and
``parenthesis'' citation \citep{blowfish}.



\section{Results}

What has been achieved in this work? 


\section{Outline}

How is the thesis structured and why? 


\section{Acknowledgments}

A big thank you for the support to \ldots 

